%!TEX root = ../main.tex

%--------------------------------------------------------%
% DOCUMENT CLASS
%--------------------------------------------------------%

  % Change "letterpaper" to "a4" if you use a4 paper size
  \documentclass[letterpaper,12pt]{article}

%--------------------------------------------------------%
% TITLE SECTION
%--------------------------------------------------------%

  %Abstract
  \usepackage{abstract} % Allows abstract customization
  % Set the "Abstract" text to bold
  \renewcommand{\abstractnamefont}{\normalfont\bfseries}
  % Set the abstract itself to small italic text
  \renewcommand{\abstracttextfont}{\normalfont\small\itshape}

  %Title
  \usepackage{titlesec} % Allows customization of titles

  %Authors
  \usepackage{authblk} % For multiple authors

  %Date
  \usepackage{datetime} % allows for including today's date
  % These two lines creates a new date format ``Month day(th), year''
  \newdateformat{usvardate}{
  \monthname[\THEMONTH] \ordinal{DAY}, \THEYEAR}

%--------------------------------------------------------%
% HEADERS & FOOTERS
%--------------------------------------------------------%

  %Footnotes
  \usepackage[bottom]{footmisc} % Makes footnotes stick to bottom of the page

  %Endnotes
  % Uncomment this line if using endnotes "\endnote{}"
  % \usepackage{endnotes}

  %Headers from page 2 on
  \usepackage{fancyhdr}
  \pagestyle{fancy}
  \fancyheadoffset{0cm}
  \setlength{\headheight}{15pt}

%--------------------------------------------------------%
% MACROS
%--------------------------------------------------------%

  % Define keywords macro command
  \providecommand{\keywords}[1]{\textbf{\textit{Keywords---}} #1}

%--------------------------------------------------------%
% MATH SUPPORT
%--------------------------------------------------------%

  % The amssymb package provides various useful mathematical symbols
  \usepackage{amssymb}
  % The amsthm package provides extended theorem environments
  \usepackage{amsthm}
  % The newtxmath package provides additional math symbol support
  % in Times New Roman symbols, etc.
  \usepackage{newtxmath}

%--------------------------------------------------------%
% FONTS
%--------------------------------------------------------%

  \usepackage{microtype} % Slightly tweak font spacing for aesthetics
  \usepackage[utf8]{inputenc}
  \usepackage{newtxtext} % Makes default font Adobe Times New Roman

%--------------------------------------------------------%
% LINES
%--------------------------------------------------------%

  % Spacing
  \usepackage{setspace} % See \doublespacing command at the top of content.tex
  % Numbering
  \usepackage{lineno,xcolor} 	% See \linenumbers at the top of content.tex

%--------------------------------------------------------%
% MARGINS
%--------------------------------------------------------%

  %NOTE: All spaces in this template are in inches, because it is
  % formatted for letterpaper (8.5 x 11 inch) paper. If you use a4
  % paper, choose different sizes in millimeters or centimeters.
  \usepackage[top=1.5in, bottom=1.5in, left=1in, right=1in]{geometry}

%--------------------------------------------------------%
% COMMENTS
%--------------------------------------------------------%

  \usepackage[colorinlistoftodos]{todonotes} % allows margin comments
  % See examples in content.tex, and here for manual:
  % http://www.ctan.org/pkg/todonotes
  \usepackage{soul} % allows for highlighting


%--------------------------------------------------------%
% ACRONYMS
%--------------------------------------------------------%

  \usepackage[nohyperlinks,printonlyused]{acronym} % Managing acronyms

%--------------------------------------------------------%
% GRAPHICS
%--------------------------------------------------------%

  \usepackage{graphicx} % More advanced figure inclusion
  \graphicspath{{figures/}} % Set the default folder for images
  \usepackage{float} % For specifying table/figure locations, i.e. [ht!]

  % The printlen command allows the user to print the exact text width or height.
  % This is useful, when trying to create graphics (outside of LaTeX, of course)
  % with the optimal dimensions. See here for usage: http://www.ctan.org/pkg/printlen
  \usepackage{printlen}

%--------------------------------------------------------%
% TABLES
%--------------------------------------------------------%

  \usepackage{longtable} % For long tables that span multiple pages
  \newcommand{\sym}[1]{\rlap{#1}}% For symbols like *** in tables
  \usepackage{tabularx} % Allows advanced table features
  \newcolumntype{L}[1]{>{\raggedright\arraybackslash}p{#1}}
  \newcolumntype{C}[1]{>{\centering\arraybackslash}p{#1}}
  \newcolumntype{R}[1]{>{\raggedleft\arraybackslash}p{#1}}
  \usepackage{relsize} % Allows precise adjustment of font size,
  %useful for fitting tables to page width

%--------------------------------------------------------%
% REFERENCES
%--------------------------------------------------------%

  \usepackage{hyperref} % For hyperlinks in the PDF
  \usepackage{csquotes}
  \usepackage[style=numeric,backend=biber,sorting=none]{biblatex}
  \bibliography{references/references.bib}
