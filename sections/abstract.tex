%!TEX root = ../main.tex

\begin{abstract}
  {
    \noindent
    Microbes tend to organize into communities consisting of hundreds of species involved in complex interactions with each other.
    16S ribosomal RNA (16S rRNA) amplicon profiling provides snapshots that reveal the phylogenies and abundance profiles of these microbial communities.
    These snapshots, when collected from multiple samples, have the potential to reveal which microbes co-occur, providing a glimpse into the network of associations in these communities.
    The inference of networks from 16S data involves numerous steps, each requiring specific tools and parameter choices.
    The extent to which the steps in this workflow affect the final network is still unclear.
    In this study, we perform a meticulous analysis of each step of a pipeline that can convert 16S sequencing data into a network of microbial associations.
    Through this process, we map how different choices of algorithms and parameters affect the co-occurrence network and estimate steps that contribute most significantly to the variance.
    We further determine the tools and parameters that generate robust co-occurrence networks and develop consensus network algorithms based on benchmarks with mock and synthetic datasets.
    The Microbial Co-occurrence Network Explorer or \acs{micone} (available at \href{https://github.com/segrelab/MiCoNE}{https://github.com/segrelab/MiCoNE}), follows these default tools and parameters and can help explore the outcome of these combinations of choices on the inferred networks.
    We envisage that this pipeline could be used for integrating multiple data sets, and for generating comparative analyses and consensus networks that can help understand microbial community assembly in different biomes.
  }
\end{abstract}

% Insert keywords here
\keywords{Microbiome, 16S rRNA, Pipeline, Interaction, Denoising, Taxonomy, Network Inference, Correlations, Qiime2, Co-occurrence, Networks, Consensus algorithm}


\section*{Importance}
  Mapping the interrelationships between different species in a microbial community is essential to understanding and controlling their structure and function.
  The surge in the high-throughput sequencing of microbial communities has led to the creation of thousands of datasets containing information about microbial abundances.
  These abundances can be transformed into co-occurrence networks, providing a glimpse into the associations within microbiomes.
  However, processing these datasets to obtain co-occurrence information relies on several complex steps, each of which involves numerous choices of tools and corresponding parameters.
  These multiple options pose questions about the robustness and uniqueness of the inferred networks.
  In this study, we address this workflow and provide a systematic analysis of how these choices of tools affect the final network, and guidelines on appropriate tool selection for a particular dataset.
  We also develop a consensus network algorithm that helps generate more robust co-occurrence networks based on benchmark synthetic datasets.
