%!TEX root = ../main.tex

\begin{abstract}
  {
    \noindent
    Microbes tend to organize into communities consisting of hundreds of species involved in complex interactions with each other.
    16S ribosomal RNA (16S rRNA) gene profiling provides snapshots that reveal the phylogenies and abundance profiles of these microbial communities.
    These snapshots, when collected from multiple samples, have the potential to reveal which microbes co-occur, providing a glimpse into the network of influences in these communities.
    The inference of networks from 16S data is prone to statistical artifacts.
    There are many tools for performing each step of the 16S analysis workflow and the results generated by different tools are varied, but the extent to which these steps affect the resultant network is still unclear.
    In this study, we perform a meticulous analysis of each step of a pipeline that processes 16S sequencing data into a network of microbial associations.
    % TODO: Instead of this say: We analyze the parts of the process that cause the most variance and what is the result
    Through this process, we determine the tools and parameters that generate the most accurate and robust co-occurrence networks.
    Ultimately, we develop a standardized pipeline that follows these default tools and parameters, but that can also help explore the outcome of any other combination of choices.
    % TODO: Make this general (do not confine it to disease-related) - niche-related(?)
    We envisage that this standard pipeline for processing 16S sequencing data into networks of microbial co-occurrences could be used for integrating multiple data-sets, and for generating comparative analyses and consensus networks useful for detecting disease-related patterns. \\
    The pipeline is available as a Python package at \href{https://github.com/dileep-kishore/mindpipe}{https://github.com/dileep-kishore/mindpipe}
  }
\end{abstract}

% Insert keywords here
\keywords{Microbiome, 16S rRNA, Pipeline, Interaction, Network, Correlations, Qiime, Co-occurrence}
