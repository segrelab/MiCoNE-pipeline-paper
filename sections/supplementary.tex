%!TEX root = ../main.tex

\newpage
\section*{Supplementary}

  \begin{table}[h]
\resizebox{\textwidth}{!}{
\begin{tabular}{lllll}
\hline
\textbf{Step}                             & \textbf{Task}                                            & \textbf{Tool}                          & \textbf{Parameter}                     & \textbf{Value}                                                                                           \\ \hline
\multirow{29}{*}{Denosing and Clustering} & \multicolumn{1}{c}{\multirow{9}{*}{Sequence Processing}} & \multirow{2}{*}{join\_reads}           & min\_overlap                           & 6                                                                                                        \\
                                          & \multicolumn{1}{c}{}                                     &                                        & perc\_max\_diff                        & 8                                                                                                        \\
                                          & \multicolumn{1}{c}{}                                     & \multirow{2}{*}{demultiplex\_illumina} & rev\_comp\_barcodes                    & False                                                                                                    \\
                                          & \multicolumn{1}{c}{}                                     &                                        & rev\_comp\_mapping\_barcodes           & False                                                                                                    \\
                                          & \multicolumn{1}{c}{}                                     & demultiplex\_454                       & -                                      & -                                                                                                        \\
                                          & \multicolumn{1}{c}{}                                     & \multirow{4}{*}{trim\_filter\_fixed}   & seq\_sample\_size                      & 10,000                                                                                                   \\
                                          & \multicolumn{1}{c}{}                                     &                                        & ncpus                                  & 1                                                                                                        \\
                                          & \multicolumn{1}{c}{}                                     &                                        & trunc\_q                               & 2                                                                                                        \\
                                          & \multicolumn{1}{c}{}                                     &                                        & max\_ee                                & 2                                                                                                        \\ \cline{2-5}
                                          & \multirow{3}{*}{Chimera Checking}                        & uchime                                 & -                                      & -                                                                                                        \\
                                          &                                                          & \multirow{2}{*}{remove\_bimera}        & ncpus                                  & 1                                                                                                        \\
                                          &                                                          &                                        & chimera\_method                        & consensus                                                                                                \\ \cline{2-5}
                                          & \multirow{17}{*}{Denoise Cluster}                        & \multirow{3}{*}{de\_novo}              & enable\_rev\_strand\_match             & True                                                                                                     \\
                                          &                                                          &                                        & suppress\_de\_novo\_chimera\_detection & True                                                                                                     \\
                                          &                                                          &                                        & ncpus                                  & 1                                                                                                        \\
                                          &                                                          & \multirow{4}{*}{closed\_reference}     & enable\_rev\_strand\_match             & True                                                                                                     \\
                                          &                                                          &                                        & suppress\_de\_novo\_chimera\_detection & True                                                                                                     \\
                                          &                                                          &                                        & ncpus                                  & 1                                                                                                        \\
                                          &                                                          &                                        & reference\_sequences                   & 97\_otus.fasta                                                                                           \\
                                          &                                                          & \multirow{5}{*}{open\_reference}       & enable\_rev\_strand\_match             & True                                                                                                     \\
                                          &                                                          &                                        & suppress\_de\_novo\_chimera\_detection & True                                                                                                     \\
                                          &                                                          &                                        & ncpus                                  & 1                                                                                                        \\
                                          &                                                          &                                        & reference\_sequences                   & 97\_otus.fasta                                                                                           \\
                                          &                                                          &                                        & picking\_method                        & uclust                                                                                                   \\
                                          &                                                          & \multirow{2}{*}{dada2}                 & ncpus                                  & 1                                                                                                        \\
                                          &                                                          &                                        & big\_data                              & FALSE                                                                                                    \\
                                          &                                                          & \multirow{3}{*}{deblur}                & ncpus                                  & 1                                                                                                        \\
                                          &                                                          &                                        & mind\_reads                            & 2                                                                                                        \\
                                          &                                                          &                                        & min\_size                              & 2                                                                                                        \\ \hline
\multirow{7}{*}{Taxonomy Assignment}      & \multirow{7}{*}{Assign}                                  & \multirow{3}{*}{naive\_bayes}          & confidence                             & 0.7                                                                                                      \\
                                          &                                                          &                                        & mem\_per\_core                         & 8G                                                                                                       \\
                                          &                                                          &                                        & ncpus                                  & 1                                                                                                        \\
                                          &                                                          & \multirow{4}{*}{blast}                 & max\_accepts                           & 10                                                                                                       \\
                                          &                                                          &                                        & perc\_identity                         & 0.8                                                                                                      \\
                                          &                                                          &                                        & evalue                                 & 0.001                                                                                                    \\
                                          &                                                          &                                        & min\_consensus                         & 0.51                                                                                                     \\ \hline
\multirow{12}{*}{OTU/ESV Processing}      & \multirow{5}{*}{Filter}                                  & \multirow{3}{*}{abundance}             & count\_thres                           & 500                                                                                                      \\
                                          &                                                          &                                        & prevalence\_thres                      & 0.05                                                                                                     \\
                                          &                                                          &                                        & abundance\_thres                       & 0.01                                                                                                     \\
                                          &                                                          & group                                  & tax\_levels                            & \begin{tabular}[c]{@{}l@{}}{[}'Phylum', 'Class', 'Order',\\ 'Family', 'Genus', 'Species'{]}\end{tabular} \\
                                          &                                                          & partition                              & -                                      & -                                                                                                        \\ \cline{2-5}
                                          & \multirow{6}{*}{Transform}                               & \multirow{6}{*}{normalize}             & count\_thres                           & 500                                                                                                      \\
                                          &                                                          &                                        & axis                                   & sample                                                                                                   \\
                                          &                                                          &                                        & prevalence\_thres                      & 0.05                                                                                                     \\
                                          &                                                          &                                        & abundace\_thres                        & 0.01                                                                                                     \\
                                          &                                                          &                                        & rm\_sparse\_obs                        & True                                                                                                     \\
                                          &                                                          &                                        & rm\_sparse\_samples                    & True                                                                                                     \\ \cline{2-5}
                                          & Export                                                   & biom2tsv                               & -                                      & -                                                                                                        \\ \hline
\multirow{17}{*}{Network Inference}       & \multirow{4}{*}{Bootstrap}                               & \multirow{3}{*}{resample}              & bootstraps                             & 1000                                                                                                     \\
                                          &                                                          &                                        & ncpus                                  & 1                                                                                                        \\
                                          &                                                          &                                        & filter\_flag                           & True                                                                                                     \\
                                          &                                                          & pvalue                                 & ncpus                                  & 1                                                                                                        \\ \cline{2-5}
                                          & \multirow{12}{*}{Correlation}                            & \multirow{2}{*}{sparcc}                & iterations                             & 50                                                                                                       \\
                                          &                                                          &                                        & ncpus                                  & 1                                                                                                        \\
                                          &                                                          & pearson                                & -                                      & -                                                                                                        \\
                                          &                                                          & spearman                               & -                                      & -                                                                                                        \\
                                          &                                                          & \multirow{5}{*}{spieceasi}             & method                                 & mb                                                                                                       \\
                                          &                                                          &                                        & ncpus                                  & 1                                                                                                        \\
                                          &                                                          &                                        & nreps                                  & 50                                                                                                       \\
                                          &                                                          &                                        & nlambda                                & 20                                                                                                       \\
                                          &                                                          &                                        & lambda\_min\_ratio                     & 1e-2                                                                                                     \\
                                          &                                                          & \multirow{2}{*}{mldm}                  & z\_mean                                & 1                                                                                                        \\
                                          &                                                          &                                        & max\_iteration                         & 1500                                                                                                     \\
                                          &                                                          & magma                                  & -                                      & -                                                                                                        \\ \cline{2-5}
                                          & Network                                                  & make\_network                          & -                                      & -                                                                                                        \\ \hline
\end{tabular}
}
\caption{The default parameters used in the various tools of the pipeline}
\label{tab:all_parameters}
\end{table}


  \renewcommand{\thefigure}{S\arabic{figure}}
  \setcounter{figure}{0}

  \begin{figure}[h]
  \centering
  \includegraphics[width=\linewidth]{figureS1.pdf}
  \caption{
    \textbf{Comparison of various denoising and clustering algorithms used in the pipeline}.
    (A, B) Correlation ofjk the abundances of the taxa that are in common between the count matrices created by two different methods.
    (A) The best correlation (most similar methods) is between open-reference and denovo.
    (B) The worst correlation (least similar methods) is between open-reference and dada2.
    (C) A heatmap showing the $\mathrm{R}^2$ of all pairwise comparisons of the methods.
  }
  \label{fig:figureS1}
\end{figure}

  \begin{figure}[h]
    \centering
    \includegraphics[width=\linewidth]{figureS2.pdf}
    \caption{
      \textbf{Heatmaps showing the weighted and unweighted unifrac distances for the hard palate dataset analysis}.
      (A) weighted unifrac distances and (B) unweighted unifrac distances between the representative sequences generated by different denoising and clustering algorithms.
      These results are in agreement with the stool microbiome dataset.
    }
    \label{fig:figureS2}
  \end{figure}

  \begin{figure}[h]
    \centering
    \includegraphics[width=\linewidth]{figureS3.pdf}
    \caption{
      \textbf{The distributions of the average weighted UniFrac distance between the expected sequence profile and the calculated sequence profile in the synthetic datasets}.
      We observe no significant difference between the various methods on the synthetic datasets used for this study.
    }
    \label{fig:figureS3}
  \end{figure}

%   \begin{figure}[h]
%     \centering
%     \includegraphics[width=0.9\linewidth]{pdf/all_denoise_reg.pdf}
%     \caption*{All pairwise correlations comparing the similarity between different denoising and clustering methods}
%     \label{fig:figureS4}
%   \end{figure}

  \begin{figure}[h]
    \centering
    \includegraphics[width=\linewidth]{figureS4.pdf}
    \caption{
      \textbf{(A)} Taxonomy composition of the 20 most abundant genera predicted for the stool microbiome dataset generated using different taxonomy references databases: Greengenes, SILVA and NCBI.
      The legend shows the common and the unique genera among the taxonomy assignments.
  }
    \label{fig:figureS4}
  \end{figure}

  \begin{figure}[h]
    \centering
    \includegraphics[width=\linewidth]{figureS5.pdf}
    \caption{
      The bray-curtis dissmilarity between the expected taxonomic composition and generated taxonomic composiion for the synthetic datasets.
  }
  \label{fig:figureS5}
  \end{figure}

  \begin{figure}[h]
    \centering
    \includegraphics[width=\linewidth]{figureS6.pdf}
    \caption{
      Calculation of presence threshold that is applied on the OTU table in the OTU processing (OP) step of the pipeline.
      This presence threhold $p_t$ is dependent on the number of samples in the dataset and the required correlation stength threshold.
  }
    \label{fig:figureS6}
  \end{figure}

  \begin{figure}[h]
    \centering
    \includegraphics[width=\linewidth]{figureS8.pdf}
    \caption{
      The similarity between the networks generated using the different network inference algorithms for stool dataset (A) and the hard palate dataset (B).
      The similarity between the various methods was found to vary with the dataset used.
  }
    \label{fig:figureS8}
  \end{figure}

  \begin{figure}[h]
    \centering
    \includegraphics[width=0.8\linewidth]{pdf/denoise_network.pdf}
    \caption{A network showing union and intersection of networks generated using certain combination of methods}
    \label{fig:figureS5}
  \end{figure}
