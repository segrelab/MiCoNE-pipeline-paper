%!TEX root = ../main.tex

\section*{Discussion}

  \subsection*{Why \ac{micone}?}

  A myriad of tools and methods have been developed for different parts of the workflow for inference of co-occurrence networks from 16S rRNA data.
  Our analyses have shown that networks generated using different combinations of tools and approaches can be substantially different from each other, highlighting the need for a clear evaluation of the source of variability and for tools that provide the most robust and accurate results.
  Our newly developed software, \ac{micone}, is a customizable pipeline for the inference of co-occurrence networks from 16S rRNA data that enables users to compare networks generated by multiple possible combinations of tools and parameters.
  Importantly, in addition to revisiting the test cases presented in this work, users will be able to explore the effect of various tool combinations on their own datasets of interest.
  The \ac{micone} pipeline has been built in a modular fashion; its plug-and-play architecture enables users to add new tools and steps, either using existing packages that have not been examined in the present work or those developed in the future.
  The \ac{micone} Python package provides functions and methods to perform a detailed analysis of the count matrices and the co-occurrence networks.
  The inferred networks are exported to a custom JSON format (see Supplementary) by default, but can also be exported to Cytoscape~\cite{shannonCytoscapeSoftwareEnvironment2003}, GML~\cite{himsoltGMLPortableGraph2010}, and many other popular formats via the Python package.

  While several tools/workflows such as \ac{qiime2}~\cite{bolyenReproducibleInteractiveScalable2019} and NetCoMi~\cite{peschelNetCoMiNetworkConstruction2020} can be used to generate co-occurrence networks from 16S sequencing data, no single tool exist that integrates the complete process of inferring microbial interaction networks from 16S sequencing reads.
  \ac{micone} is unique as it offers this functionality packaged in a workflow that can be run locally, on the compute cluster, or in the cloud.

  \subsection*{The default pipeline and recommended tools}

  Through \ac{micone}, in addition to transparently revealing the dependence of co-occurrence networks on tool and parameter choices (see Discussion in Supplementary Text for details on the DC, TA and OP steps), we have taken advantage of our spectrum of computational options and the availability of mock and synthetic datasets, to suggest a default standard setting.
  Additionally, we have developed a consensus approach, that can reliably generate networks that are fairly robust across multiple tool choices.
  An important caveat related to these results is that due to the lack of a universal standard for microbial interaction data, our conclusions are based on the specific datasets used in our analysis.
  While our analysis is based on several mock and synthetic datasets that cover a diverse range of abundance distributions and network topologies, datasets that have drastically different distributions may require a re-assessment of the best settings.

  The networks generated by different network inference methods show considerable differences in edge-density and connectivity, partially due to the underlying assumptions regarding sparsity, distribution, and compositionality.
  To address this issue, we have developed two consensus algorithms (simple voting and scaled-sum method) that generate networks whose links have evidence based on multiple inference algorithms.

 We find that the scaled-sum method performs the best on synthetic datasets, and is therefore chosen as the default for the \ac{ni} step of the pipeline.
 Notably, the consensus network displays a higher precision and returns a concise list of robust associations which represent a valuable set for experimental validation follow-up.

  \subsection*{Future directions}

  Future work building upon our current results could enhance the network inference process in multiple ways.
  The current analyses make use of one fecal microbiome transplant dataset with healthy and ASD samples, three mock community datasets, and several datasets generated by two synthetic interaction methods.
  Incorporating datasets from a broad spectrum of biomes with varying microbial distributions into \ac{micone} will likely increase the robustness and generalizability of the results from these analyses.

  The network analyses in this study are primarily at the Genus level, wherein the lowest resolution of a node is a Genus and if an entity cannot be resolved to the Genus level, the next lowest taxonomic level is used (for example, Family).
  As a consequence, two entities belonging to the same lineage where one entity is resolved to the Genus level and another is resolved to the Family level are treated as two different nodes in the network.
  Thus, the development of a metric of overlap to compare nodes with shared lineages within and across networks could enable more biologically and phylogenetically relevant comparisons.

  Although direct comparisons between co-occurrence networks and directly measured interactions are difficult to interpret and highly debated~\cite{hiranoDifficultyInferringMicrobial2019,gobernaCautionaryNotesUse2022}.
  Further, benchmarking of co-occurrence networks could also be pursued through the use of literature-based interactions~\cite{lima-mendezDeterminantsCommunityStructure2015a} or biological benchmark interaction data~\cite{sungGlobalMetabolicInteraction2017a}.
  Additionally, \ac{micone} could be extended to enable the processing of metagenomics sequencing data, facilitating the analysis of a much larger and diverse range of datasets and domains of life.

  Although in the current analysis, we have only used default parameter values recommended by the tool creators, the \ac{micone} pipeline could be used in the future to explore any combinations of parameters and to optimize these values for improved network inference.
  Overall, there likely is no ``best method'' for the various steps of 16S data analysis, and hence, \ac{micone} is intended to help researchers to identify the methods and algorithms that are most suitable for their datasets in an easy-to-use and reproducible manner.

  We envision that \ac{micone}, and its underlying tools and databases, will be increasingly useful for building large comparative analyses across studies.
  It enables rapid, configurable, and reproducible inference of microbial networks and furthers the formulation of hypotheses about the role of these interactions on community composition and stability.
  These comparative analyses will require coupled network analysis and visualization tools (such as \ac{mind}~\cite{huResourceComparisonIntegration2022}) and need systematic access to datasets, shared in accordance with FAIR standards~\cite{pachecoFAIRRepresentationsMicrobial2022}.
