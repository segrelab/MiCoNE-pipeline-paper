%!TEX root = ../main.tex

\section*{Introduction}

  Microbial communities are ubiquitous and play an important role in biogeochemical transformations in both natural as well as man-made ecosystems.
  These microbiomes comprise several thousand different microbial strains interacting with each other and their environment, often through intricate metabolic and signaling relationships.
  The changes in the microbiome can impact the environmental niche in which they reside \cite{HumanMicrobiomeProjectConsortium2012,Lloyd-Price2016} and for host-associated microbiomes this can have major implications in health of the human body.
  A mechanistic understanding of the key interactions between microbes is necessary in order to analyze the dysbiosis in a healthy microbiome and the design interventions.

  Studies have shown the importance of specific microbial interactions in the healthy microbiome \cite{Lloyd-Price2016} and others have shown how changes in these interactions can lead to dysbiosis \cite{Wang2017,Gilbert2016,Belizario2015}.
  While direct measurement of interactions is challenging, a significant effort has gone into measuring correlations leading to the correlations between microbial abundances being readily available(?)
  Co-occurrence relationships in microbial communities may be driven by microbial interactions, although the extent and nature of their relationship is highly debated.
  Each interaction within this complex ecological network can have a positive effect, a negative effect or no effect on the species involved.
  Bacteria that compete for the same nutrient will have a negative co-occurrence relationship \cite{Ghoul2016} whereas cross-feeding between species leads to a positive co-occurrence relationship \cite{DSouza2018}.
  The microbial interaction networks are highly dynamic, and they constantly reorganize in a varying environment.
  Even under static conditions, these interactions are often difficult to predict due to their non-linear nature \cite{Konopka2015}.
  Compounding on this complexity is the fact that these interaction networks usually involve thousands of species.

  High-throughput metagenomic sequencing techniques (such as 16S rRNA gene profiling) help detect, identify and quantify a large part of the constitutive microorganisms of a microbiome \cite{Jovel2016}.
  Knowledge of the community composition at a particular instance in time would enable us to derive partial insights into the underlying dynamics.
  With the advancement in DNA sequencing technologies \cite{Narihiro2017} and data processing methods much more information can be extracted from these microbial community samples than ever before.
  These advances have led to large-scale data collection efforts involving soil (Earth Microbiome Project) \cite{Thompson2017}, marine (Tara Oceans Project) \cite{Zhang2015} and human-associated microbiota (Human Microbiome Project) \cite{HumanMicrobiomeProjectConsortium2012} and thus led the generation of a wealth of data about these microbial communities.
  These metagenomic data-sets can be used to infer networks of co-occurrence relationships between microorganisms in the samples.
  The co-occurrence networks can in turn be used to identify important interactions driving the dynamics of the microbial community, as well as, analyze emergent properties such as robustness and modularity.
  However, 16S sequencing data is limited by factors such as resolution, sequencing depth, compositional nature, sequencing errors and copy number variations.
  To counter these limitations and remove statistical artifacts, many different analysis methods have been developed \cite{Callahan2016,Amir2017,Friedman2012,Kurtz2015}.
  But, the different methods and tools developed to solve these issues infer vastly different community compositions and co-occurrence networks \cite{Golob2017,Weiss2016}.

  The methodology for the analysis of 16S rRNA sequence data is not very straightforward and there are a myriad of tools for each step of the process, making the analysis complicated and not very reproducible.
  But because there does not exist a dataset wherein the microbial interactions have been fully characterized, there is no straightforward way to determine which set of tools or methods generate the most accurate networks.
  In this study, we propose to build a standardized 16S data-analysis pipeline that will produce robust and reproducible co-occurrence networks from community 16S sequence data.
  We do this by conducting a systematic study of the various methods currently available.
  We compare different methods to each other and additionally use mock microbial community datasets and synthetically generated reads to benchmark them.

  There exist many platforms like MG-RAST \cite{Keegan2016}, Qiita \cite{qiita} and tools such as QIIME \cite{Caporaso2010} that provide pipelines for 16S data analysis, but the main motivations behind the construction of our own pipeline are two-fold.
  Firstly, the existing tools do not focus their analysis on co-occurrence networks and the effect of upstream statistical methods on the inferred networks.
  Secondly, we undertake a comprehensive study of existing methods in order to suggest optimal methods, tools and parameters for users of the pipeline.
  We believe that these steps will ensure better reproducibility and easier comparison of networks across data-sets.
